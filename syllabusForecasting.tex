\documentclass[12pt, margin=.5in]{article}

\usepackage{fontspec}
\usepackage{hyperref}

\defaultfontfeatures{Mapping=tex-text}
\setmainfont {Adobe Garamond Pro} % Main document font
\setsansfont {Gill Sans} % Used in the from address line above the to address


\pagestyle{empty}

\begin{document}
\vspace*{-6em}
\begin{center}
{\Large ECON 8310 (BSAD 8080)\   \ -- \ Business Forecasting \\[.5em] Date \& Time \ -- \   Room: TBA   
}
\end{center}

\setlength{\unitlength}{1in}

\hspace*{-4em}\begin{picture}(6,.1) 
\put(0,0) {\line(1,0){6.25}}         
\end{picture}

 

\renewcommand{\arraystretch}{2}

\begin{itemize}
\vskip.25in
\item[\textbf{Instructor:}] Dustin White\\  MH 332M\\ Phone: 402-554-3303
\vskip.25in
\item[\textbf{Office Hours:}] TBA, and by appointment.

\vskip.25in
\item[\textbf{Materials:}] Course Slides (hosted on Blackboard)\\ Course Notes (also hosted on Blackboard)\\ Python (I recommend Anaconda, since it comes prepackaged with most of the numeric and analytic libraries we will use)

\vskip.25in
\item[\textbf{Prerequisites:}]
ECON 2200, BSAD 8150, or equivalent\\
BSAD 2130 or equivalent\\

No previous programming experience is required, but some experience programming is highly advantageous. We will be creating our own forecasts, and my examples will use Python. You are welcome to use any other statistical software (such as R), but \textbf{I will not provide support for other software or languages}.

\vskip.25in
\item[\textbf{Description:}]
The course will cover forecasting tools and applications applied to business settings. We will cover traditional Econometric forecasting methods in the first half of the class. In the second half of the course, we will focus on models in predictive analytics and machine learning, since these models are quickly becoming critical tools for forecasters in many settings. The course will include lecture and lab time, and labs will be focused on teaching students how to implement the models discussed in lectures.


\vspace*{.15in}
\item[\textbf{Course Outline:}]

\textbf{Time Series Models}\hfill \\
Review of OLS, Tools for Class \dotfill 1 day\\
Time Series Models - ARIMA and ARIMAX \dotfill approx 1 day\\
Time Series Models - VAR \dotfill approx 1 day\\
\textbf{Predictive Models}\hfill \\
Classification and Naive Bayes \dotfill approx 1 day\\
Entropy, Histograms and Decision Tree Classifiers \dotfill approx 2 days\\
Support Vector Machines \dotfill approx 1 day\\
Midterm Exam \dotfill 1 day\\
Ensemble Methods \dotfill approx 2 days\\
Neural Networks - Introduction \dotfill approx 1 day\\
Neural Networks - Deep Neural Nets \dotfill approx 1 day\\
Simulation \dotfill approx 2 days\\
Final Exam \dotfill 2 days\\



\vspace*{.15in}
\item[\textbf{Grade Policy:}] 
Lab Completion \dotfill 300 points\\
Midterm Exam \dotfill 100 points \\
Final Exam \dotfill 100 points \\
Project 1 \dotfill 250 points\\
Project 2 \dotfill 250 points

\vspace*{1em}
Grades will be distributed according to the following grade scale: \\

\begin{tabular}{l|l|l|l}
Score & Letter Grade & Score & Letter Grade\\
\hline
A & > 939 & C+ & 775 - 799 \\
A- & 900 - 939 & C & 725 - 774 \\
B+ & 875 - 899 & C- & 700 - 724 \\
B & 825 - 874 & D & 600 - 699 \\
B- & 800 - 824 & F & < 600 \\
\end{tabular}


\vskip.25in
\item[\textbf{Course Objectives}:] After this course, students should be capable of:
\begin{enumerate}
\item Understanding the respective strengths and weaknesses of the models presented in class
\item Implementing Econometric forecasting models
\item Applying machine learning algorithms in a forecasting setting 
\end{enumerate}

\vskip.25in
\item[\textbf{Grading}:] All assignments are to be submitted through the appropriate dropboxes on the course website. Rubrics will be posted, and will contain detailed information on the assignment grading policy. 

\vskip.25in
\item[\textbf{Homework}:]  Late work is not accepted, except as outlined in University policy.

\vskip.25in
\item[\textbf{Academic Honesty}:]  UNO’s requirements for Academic Integrity and Behavior All students are required to adhere to the highest standards of academic integrity and behavior and must satisfy the UNO Academic Integrity Policy \href{http://www.unomaha.edu/student-life/student-conduct-and-community-standards/policies/academic-integrity.php}{www.unomaha.edu/student-life/student-conduct-and-community-standards/policies/academic-integrity.php} and Student Code of Conduct \href{http://www.unomaha.edu/student-life/student-conduct-and-community-standards/policies/code-of-conduct.php}{www.unomaha.edu/student-life/student-conduct-and-community-standards/policies/code-of-conduct.php}. It is the student’s responsibility to read, understand and abide by these policies. If I find that you have plagiarized, been dishonest in completing your assignments, or cheated an an exam or assignment, then I reserve the right to award you no points on the entire exam, project, or assignment and to report the behavior to the university. If this behavior is repeated, I reserve the right to award a failing grade, independent of your score on other assignments. Academic integrity is essential to education, and I take it very seriously.

\vskip.25in
\item[\textbf{Extra Help}:]  Dot not hesitate to come to my office during office hours or by appointment
to discuss a homework problem or any aspect of the course. 



\end{itemize}







\end{document}